\chapter{Conclusion and Future Scope}

This project presented a complete pipeline to transform a photographed business card into laser-ready G-code within the context of the Digital Factory. By integrating computer vision, OCR, vector graphics, rasterisation, G-code generation, and agent-based orchestration with LangGraph, the system demonstrated how a traditionally manual process can be automated and streamlined.
The implementation showed that:
\begin{itemize}
	\item Information extraction and layout preservation were achieved using a vision-language model (Qwen2.5-VL) and OCR.
\item SVG generation and LLM-assisted editing enabled both automation and human-in-the-loop refinement, balancing efficiency with control.
\item Rasterisation and binarisation produced engraving-ready bitmaps, ensuring clarity and contrast.
\item Scanline G-code generation with preview translated designs into precise toolpaths, with user validation reducing risk of errors.
\item The LangGraph orchestration provided modularity, robustness, and scalability, allowing individual agents to work independently while contributing to a coherent workflow.

\end{itemize}

Overall, the project validated its requirements: the pipeline proved capable of automatically generating engraving instructions, with opportunities for user-driven refinements, thereby enhancing the functionality of the Laser Engraver Module in an Industry 4.0 setting.


