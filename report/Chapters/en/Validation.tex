\chapter{Validation of the Requirements and Limitations}

\section{Validation}
This section validates how each project requirement was fulfilled through the implementation and testing processes.

\subsection{Information Extraction}
\begin{itemize}
	\item The OCR Agent successfully extracted structured text (name, role, email, phone) using Qwen2.5-VL via Fireworks API.
	\item The Visual Analysis Agent identified layout elements such as logos and QR codes with bounding boxes mapped to millimetre coordinates (85×54 mm).
	\end{itemize}


\subsection{Layout Preservation}
\begin{itemize}
	\item The SVG Agent embedded extracted content into a vector file, preserving relative positions and scales.
	The Y-axis inversion (from bottom-left to SVG’s top-left) was handled consistently, ensuring fidelity to physical card proportions.
	\item Validation involved overlaying bounding boxes on the generated SVG and comparing them with original card layouts, demonstrating positional accuracy consistent.
\end{itemize}

\subsection{SVG Generation and Editing}
\begin{itemize}
	\item The system produced initial SVG files automatically, which could then be refined using either command-based editing or natural language editing (via the LLM-SVG Agent).
	Human-in-the-loop testing confirmed that edits such as “Replace 'Yaman Alsaady' with 'Vatsal Mahajan'.” or “add\_text tagline at x=3 y=6 text='Created by Vatsal' size=3.2” were correctly parsed and applied.
\end{itemize}

\subsection{SVG Generation and Editing}
\begin{itemize}
	\item The system produced initial SVG files automatically, which could then be refined using either command-based editing or natural language editing (via the LLM-SVG Agent).
	\item Human-in-the-loop testing confirmed that edits such as “Replace 'Yaman Alsaady' with 'Vatsal Mahajan'.” or “add\_text tagline at x=3 y=6 text='Created by Vatsal' size=3.2” were correctly parsed and applied.
\end{itemize}

\subsection{Rasterisation and Binarisation}
\begin{itemize}
	\item The Rasterisation Module (CairoSVG + OpenCV) converted SVGs into high-contrast bitmaps, ensuring readability for raster scanline engraving.
	\item Empirical tests showed that thresholding removed background noise and preserved fine details such as text edges and QR code readability.
\end{itemize}

\subsection{G-code Generation and Preview}
\begin{itemize}
	\item The G-code Agent implemented a zig-zag scanline algorithm, minimizing non-productive travel while maintaining engraving precision.
	\item The G-code Preview Agent displayed the toolpaths, enabling error detection before physical execution.
	\item Comparative checks between preview and engraved outputs confirmed high fidelity, in line with G-code raster engraving principles.
\end{itemize}

\section{Limitation}
Although the system fulfils its primary objective of generating laser-ready G-code from business card images, several limitations remain:
\begin{itemize}
	\item When logos and text are embedded within the same region, the system prioritises the text, often leaving the logo unprocessed. This limits the accuracy of reproducing complex card layouts.
	\item The vision model is not explicitly trained for arbitrary logo recognition. Logos are only reproduced if they already exist in the system’s database, restricting generalisation to unseen designs.
	\item In cases where logos are retrieved from the database, the system writes them into the SVG with reduced size, which can distort the intended visual balance of the design.
	\item The model successfully detects the position and size of QR codes, it does not render them into the SVG due to security restrictions.
	\item Both the large language model (LLM) and vision-language model (VLM) currently run on free credits. Once these are exhausted, continued operation requires subscription costs, which may affect scalability.
	\item The system relies on openly available, free versions of LLMs and VLMs, which show limited accuracy compared to commercial or fine-tuned models, particularly for detailed layout extraction.
	\item The editing interface interprets only a narrow set of commands via regex matching (e.g., move, delete, replace, add\_text). This restricts the naturalness of user interactions and may limit usability in broader contexts.
\end{itemize}

